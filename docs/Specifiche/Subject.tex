\chapter{Subject}
  \section{Descrizione}
    Questo è il firmware che verrà eseguito sui \textit{nodi sensore}, i quali si comporteranno come dei CoapServer, avviando un server rest, in ascolto sulla porta di default
    di CoAP 5683, gestendo le richieste ricevute dal \textit{Proxy}, che possono essere:
    \begin{itemize}
      \item Discovery delle risorse presenti sul nodo
      \item Registrazione ad una risorsa con uno specifico tipo di priorità
      \item Cancellazione dall'osservazione di una risorsa
    \end{itemize}
    Il nodo sensore, solo nel caso in cui è registrato almeno un \textit{Observer} ad una risorsa, si occuperà di eseguire il sensing e l'invio di quest'ultima nel caso in cui
    il valore sia cambiato o il vecchio valore stia per scadere (questo secondo caso è gestito per evitare di perdere la registrazione dell'\textit{Osservatore}).\newline
    Il nodo sensore si occupa anche di connettersi e di mantenere viva la connessione col \textit{Border Router}, utilizzato dal \textit{Proxy} per comunicare con i nodi sensore.

  \section{Modifiche al Border Router}
    Per poter compilare il codice del border router fornito con contikiOS per dispositivi di tipo SkyMote è stato necessario disabilitare il processo che esegue il web server in quanto il dispositivo non dispone di
    memoria sufficiente per poter utilizzare questa funzione; la quale veniva utilizzata semplicemente con lo scopo di far vedere all'utente le rotte presenti al momento e la lista dei vicini tramite interfaccia web,
    quindi rimuovendo questo processo non vengono rimosse funzionalità necessarie ad un router.\newline
    Per rimuovere la funzionalità è bastato ridefinire WEBSERVER uguale a 0.
    \lstinputlisting[label={BorderRouter}, language=C, firstnumber=64 ,firstline=64, lastline=67, caption={BorderRouter, \href{https://tinyurl.com/yxzcal2z}{codice originale}},captionpos=b]{../Codice/C/border-router.c}

  \section{Implementazione}
    \subsection{Process Rest Server}
      Processo principale dei nodi sensore, si occupa di:
      \begin{enumerate}
        \item Ottenere un indirizzo IP connettendosi al border-router
        \item Rendere disponibili le risorse presenti attivandole
        \item Attivare i sensori relativi alle singole risorse
      \end{enumerate}


      \lstinputlisting[label={RestServer}, language=C, firstnumber=38, firstline=38, caption={Process RestServer},captionpos=b]{../Codice/C/subscriber.c}

    \subsection{Risorse}
      Per avere una maggiore modularità è stato realizzato un file per ogni singola risorsa e risorse sono state definite come variabili di tipo \textbf{extern}
      \lstinputlisting[label={ExternResources}, language=C, firstnumber=4, firstline=4, lastline=12, caption={Risorse presenti sul nodo},captionpos=b]{../Codice/C/subscriber.c}

      \subsubsection{Batteria}
        La batteria viene fornita come risorsa di tipo osservabile, l'unico osservatore che si iscrive per ricevere le notifiche relative a questa risorsa è il \textit{Proxy},
        in quanto se la batteria del sensore si trova sotto una specifica soglia vengono accettate soltanto richieste con un livello di priorità critico.\newline
        Il sensing del livello della batteria viene fatto periodicamente
        Le notifiche della batteria vengono inviate solo in due casi:
        \begin{enumerate}
          \item Valore della batteria è sotto la soglia critica
          \item L'ultimo valore sta per scadere
        \end{enumerate}


      \subsubsection{Temperatura}
