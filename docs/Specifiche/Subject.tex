\chapter{Subject}
  \section{Descrizione}
    Questo è il firmware che verrà eseguito sui \textit{nodi sensore}, i quali si comporteranno come dei CoapServer, avviando un server rest, in ascolto sulla porta di default
    di CoAP 5683, gestendo le richieste ricevute dal \textit{Proxy}, che possono essere:
    \begin{itemize}
      \item Discovery delle risorse presenti sul nodo
      \item Registrazione ad una risorsa con uno specifico tipo di priorità
      \item Cancellazione dall'osservazione di una risorsa
    \end{itemize}
    Il nodo sensore, solo nel caso in cui è registrato almeno un \textit{Observer} ad una risorsa, si occuperà di eseguire il sensing e l'invio di quest'ultima nel caso in cui
    il valore sia cambiato o il vecchio valore stia per scadere (questo secondo caso è gestito per evitare di perdere la registrazione dell'\textit{Osservatore}).\newline
    Il nodo sensore si occupa anche di connettersi e di mantenere viva la connessione col \textit{Border Router}, utilizzato dal \textit{Proxy} per comunicare con i nodi sensore.

  \section{Modifiche al Border Router}
    Per poter compilare il codice del border router fornito con contikiOS per dispositivi di tipo SkyMote è stato necessario disabilitare il processo che esegue il web server in quanto il dispositivo non dispone di
    memoria sufficiente per poter utilizzare questa funzione; la quale veniva utilizzata semplicemente con lo scopo di far vedere all'utente le rotte presenti al momento e la lista dei vicini tramite interfaccia web,
    quindi rimuovendo questo processo non vengono rimosse funzionalità necessarie ad un router.\newline
    Per rimuovere la funzionalità è bastato ridefinire WEBSERVER uguale a 0.
    \lstinputlisting[label={BorderRouter}, language=C, firstnumber=64 ,firstline=64, lastline=67, caption={BorderRouter, \href{https://tinyurl.com/yxzcal2z}{codice originale}},captionpos=b]{../Codice/C/border-router.c}

  \section{Implementazione}
    \subsection{Process Rest Server}
      Processo principale dei nodi sensore, si occupa di:
      \begin{enumerate}
        \item Ottenere un indirizzo IP connettendosi al border-router
        \item Rendere disponibili le risorse presenti attivandole
        \item Attivare i sensori relativi alle singole risorse
      \end{enumerate}

      %A seconda di come viene usato e se viene usato o meno alla fine del progetto, inserisci qualche nota relativa a powertrace

      \lstinputlisting[label={RestServer}, language=C, firstnumber=56, firstline=56, caption={Process RestServer},captionpos=b]{../Codice/C/subscriber.c}

    \subsection{Risorse}
      Per avere una maggiore modularità è stato realizzato un file per ogni singola risorsa e risorse sono state definite come variabili di tipo \textbf{extern}

      \lstinputlisting[label={ExternResources}, language=C, firstnumber=8, firstline=8, lastline=12, caption={Risorse presenti sul nodo},captionpos=b]{../Codice/C/subscriber.c}

      Per il nostro scopo vengono utilizzate esclusivamente risorse osservabili, che devono essere dichiarate come risorse di tipo \textbf{PERIODIC\_RESOURCE};
      la peculiarità di questo tipo di risorse è il fatto che al momento dell'inizializzazione sono necessarie:
      \begin{itemize}
        \item get\_handler: una funzione utilizzata nel caso in cui venga fatta una richiesta di tipo get alla risorsa (necessaria anche per risorse non periodiche),
              ma che viene richiamata, anche, ogni volta sia necessario inviare un nuovo valore ai vari osservatori
        \item Il periodo con il quale viene richiamata la funzione periodic\_handler
        \item periodic\_handler: una funzione che viene utilizzata per eseguire il sensing della risorsa, verificare se è necessario o meno inviare il valore ottenuto, a causa
              della scandenza della vecchia risorsa oppure della rilevazione di un valore critico o meno che differisce dal precedente di una certa soglia, e in caso di invio
              decide anche il valore del campo MaxAge
      \end{itemize}
      Le risorse utilizzate ed offerte dal nodo sono:
      \begin{itemize}
        \item Batteria
        \item Temperatura
        \item Luminosità
        \item Umidità
      \end{itemize}
      Per permetterci di eseguire la fase di Testing, tutte le risorse sono simulate in quanto il Testing verrà eseguito in un ambiente stabile (non ci saranno variazioni
      significative di temperatura/luminosità/umidità).\newline
      Più osservatori possono iscriversi per ricevere le notifiche relative a queste risorse, fatta eccezione della batteria; la registrazione da parte di più osservatori
      viene gestita dal \textit{Proxy}, come descritto sopra, in modo tale che al \textit{subscriber} arrivi una sola richiesta di registrazione (quella relativa alla registrazione
      da parte del Proxy).\newline
      Durante la fase di registrazione viene gestito il tipo dei messaggi che si vuole ricevere \textbf{CRITICAL} (solo messaggi di tipo critico), o \textbf{NON\_CRITICAL} (tutti i
      messaggi critici e non).\newline
      Quando una notifica viene inviata gli viene assegnato un \textit{Tempo di validità} indicato all'interno del campo \textbf{MaxAge} del pacchetto; il \textit{Subscriber} si
      preoccupa di inviare una nuova notifica sempre prima dello scadere della precedente notifica, che sia con un nuovo valore o con l'ultimo valore che ha inviato, questo
      per evitare di dover rieseguire tutto il protocollo di registrazione alla risorsa da parte del \textit{Proxy} e degli \textit{Observer}.

      \subsubsection{Batteria}
        La batteria viene fornita come risorsa di tipo osservabile, l'unico osservatore che si iscrive per ricevere le notifiche relative a questa risorsa è il \textit{Proxy},
        in quanto se la batteria del sensore si trova sotto una specifica soglia (\textbf{CRITICAL\_BATTERY}) vengono accettate
        soltanto richieste con un livello di priorità critico.\newline
        Il livello della batteria è simulato e viene fatto variare in base al tipo di operazione (sensing e trasmissione del valore)
        di una certa quantità costante differente per ogni tipologia (\textbf{SENSING\_DRAIN}, \textbf{TRANSMITTING\_DRAIN}).\newline
        Il sensing del livello della batteria viene fatto periodicamente.Invece per quanto riguarda le notifiche della batteria vengono inviate solo in due casi:
        \begin{enumerate}
          \item Valore della batteria è sotto la soglia critica
          \item L'ultimo valore sta per scadere, per evitare che il protocollo per la registrazione alla batteria venga rieseguito nuovamente
        \end{enumerate}

        Una volta che la batteria è stata inviata perché la soglia è scesa sotto il livello critico, questa non verrà più inviata al \textit{Proxy}, in quanto ormai, quest'ultimo,
        gestirà le richieste di registrazione in modo opportuno e nel momento in cui non riceverà più valori da quel nodo, saprà che la batteria è scesa a 0.

        \lstinputlisting[label={res-battery}, language=C, caption={Codice di gestione relativo alla risorsa batteria},captionpos=b]{../Codice/C/resources/res-battery.c}


      \subsubsection{Temperatura}
        La temperatura è una risorsa di tipo osservabile, un valore è considerato critico nel caso in cui è superiore ad una soglia (\textbf{TEMPERATURE\_CRITICAL\_THRESHOLD})
        e viene inviato solo nel caso in cui sia la differenza con il valore precedentemente inviato sia di almeno \textbf{TEMPERATURE\_CRITICAL\_CHANGE},
        o sia scaduto il tempo di validità dell'ultima notifica inviata relativa a questa risorsa; invece il valore viene considerato non critico nel caso in cui sia sotto la
        soglia critica e viene inviato subito se differisce dal vecchio valore di una quantità maggiore di \textbf{TEMPERATURE\_NON\_CRITICAL\_CHANGE}, altrimenti poco prima
        dello scadere del tempo di validità

        \lstinputlisting[label={res-battery}, language=C, caption={Codice di gestione relativo alla risorsa temperatura},captionpos=b]{../Codice/C/resources/res-temperature.c}

      \subsubsection{Luminosità}
        La luminosità è una risorsa di tipo osservabile, un valore è considerato critico solo nel caso in cui la differenza col valore precedentemente inviato sia di almeno \textbf{LUMINOSITY\_CRITICAL\_CHANGE}, altrimenti viene considerato come valore non critico da inviare se la differenza è minore alla soglia critica, ma maggiore
        di almeno \textbf{LUMINOSITY\_NON\_CRITICAL\_CHANGE}; nel caso di scadenza del tempo di validità il valore è considerato sempre di tipo non critico e inviato
        solo nel caso la registrazione alla risorsa sia per tutti i tipi di messaggio e non solo per i critici

        \lstinputlisting[label={res-battery}, language=C, caption={Codice di gestione relativo alla risorsa luminosità},captionpos=b]{../Codice/C/resources/res-luminosity.c}

      \subsubsection{Umidità}
        L'umidità è una risorsa di tipo osservabile, un valore è considerato non critico nel caso in cui questo si trovi all'interno dell'intervallo dei valori non critici,
        i valori compresi tra \textbf{HUMIDITY\_CRITICAL\_MIN\_THRESHOLD} e \textbf{HUMIDITY\_CRITICAL\_MAX\_THRESHOLD}, viene inviato subito nel caso in cui differisca dal
        valore precedente di un valore superiore a \textbf{HUMIDITY\_NON\_CRITICAL\_CHANGE}, altrimenti si attende il termine del tempo di validità della notifica precedente;
        tutti gli altri valori sono considerati critici e inviati subito nel caso in cui la differenza col valore precedente sia superiore a \textbf{HUMIDITY\_CRITICAL\_CHANGE}

        \lstinputlisting[label={res-battery}, language=C, caption={Codice di gestione relativo alla risorsa umidità},captionpos=b]{../Codice/C/resources/res-humidity.c}

    \subsection{Parametri}
