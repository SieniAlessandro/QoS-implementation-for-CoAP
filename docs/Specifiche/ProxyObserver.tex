\chapter{Modulo ProxyObserver}
	\section {Descrizione}
	\section {Implementazione}
		\subsection {Modifiche a Californium lato Server}
			Al fine di implementare le modifiche effettuate al protocollo CoAP, è stato necessario modificare anche la libreria Californium. In seguito verranno elencati i cambiamenti al codice della libreria, indicando il nome del file con un link alla repository con il codice originale e la motivazione di tale modifica. I file fanno riferimento alla package core di Californium:
			\subsubsection{coap.CoAP}
				Il livello di priorità viene indicato usando i primi 2 bit, rinominato campo \textit{QoS} del campo Observe del pacchetto CoAP ( 3 bytes ).  Questa classe fornisce i valori che questo campo può assumere\newline
				\lstinputlisting[language=Java, firstline=663, lastline=673, caption={coap.CoAP.QoSLevel, \href{https://tinyurl.com/y46rbbg6}{codice originale}},captionpos=b]{../Codice/Java/CoAP.java}

			\subsubsection{coap.Request}
				Nel protocollo standard una Observe Request Registration può avere solo il valore 0 nel campo \textit{Observe}. Nel caso considerato invece i valori possono essere 4, corrispondenti ai 4 livelli di priorità. Questa funzione deve ritorna \textit{True} se 	contiene uno di questi valori. \newline
				\lstinputlisting[language=Java, firstline=636, lastline=643, caption={coap.Request, \href{https://tinyurl.com/y2lhtgo4}{codice originale} },captionpos=b]{../Codice/Java/Request.java}

			\subsubsection{observe.ObserveRelation}
			Una ObserveRelation mantiene le informazioni relative alla relazione tra un observer e la risorsa osservata. \'E stato aggiunto un 	campo alla classe per mantenere il livello di priorità di quella relazione, i suoi get e set e una funzione per comparare 2 relazioni in base a questo campo. Quest'ultima servirà nel meccanismo di notifica degli observer basato su priorità. \newline
				\lstinputlisting[language=Java, firstline=61, lastline=63, caption={observe.ObserveRelation QoS, \href{https://tinyurl.com/y3rf5qgl}{codice originale}},captionpos=b]{../Codice/Java/ObserveRelation.java}
				\lstinputlisting[language=Java, firstline=130, lastline=147, caption={observe.ObserveRelation funzioni QoS, \href{https://tinyurl.com/y3rf5qgl}{codice originale}},captionpos=b]{../Codice/Java/ObserveRelation.java}

			\subsubsection{server.ServerMessageDeliverer}
				Quando una \textit{ObserveRelation} viene creata in seguito alla ricezione di una richiesta con l’opzione Observe, il campo \textit{QoS} della relazione viene settato con il valore del campo Observe della richiesta. \newline
				\lstinputlisting[language=Java, firstline=153, lastline=165, caption={server.ServerMessageDeliverer checkForObserveOption(...), \href{https://tinyurl.com/y3tom2xy}{codice originale}},captionpos=b]{../Codice/Java/ServerMessageDeliverer.java}

			\subsubsection{server.ServerState}
				Enumerato che definisce i 3 stati del nodo sensore:
				\begin{enumerate}
					\item \textbf{UNAVAILABLE}: il nodo sensore non è attivo, qualsiasi richiesta relativa a quel sensore è rigettata.
					\item \textbf{ONLY\_CRITICAL}: il nodo sensore ha un autonomia al di sotto di una certa soglia, quindi si accettano solo richieste da parte di observe richiedenti solo gli eventi critici della risorsa.
					\item \textbf{AVAILABLE}: il nodo non ha problemi di autonomia quindi è possibile accettare qualsiasi tipo di richiesta.
				\end{enumerate}
				\lstinputlisting[language=Java, firstline=3, lastline=5, caption={ServerState},captionpos=b]{../Codice/Java/ServerState.java}

			\subsubsection{CoapResource}
				\paragraph{Aggiornamento relazioni dopo il cambio stato del sensore}
				Quando avviene un cambio di stato di un sensore, è necessario controllare che le relazioni già stabilite siano consistenti con il nuovo stato. Pertanto quando lo stato diventa \textbf{ONLY\_CRITICAL}, le relazioni con livello \textit{CoAP.QoSLevel.NON\_CRITICAL\_MEDIUM\_PRIORITY} e \textit{CoAP.QoSLevel.NON\_CRITICAL\_LOW\_PRIORITY} vengono cancellate, mentre se si passa ad \textbf{UNAVAILABLE}, tutte le relazioni di quella risorsa vengono cancellate. \'E stata quindi aggiunta una funzione per cancellare solo le relazioni con un livello non critico di priorità. \newline
				\lstinputlisting[language=Java, firstline=557, lastline=567, caption={CoapResource clearAndNotifyNonCriticalObserveRelations, \href{https://tinyurl.com/y264lheh}{codice originale}},captionpos=b]{../Codice/Java/CoapResource.java}
				\paragraph{Notifica basata su priorità}
				È stata aggiunta una lista \textit{sortedObservers} di ObserveRelation ordinata in base al campo \textit{QoS} di un ObserveRelation in modo decrescente. L’ordinamento è effettuato all’inserimento di una nuova relazione nel \textit{ObserveRelationContainer observeRelations}. Questa nuova lista è impiegata per effettuare l’invio delle notifiche in ordine di priorità: l'invio delle notifiche è effettuato scansionando in modo sequenziale questa lista, partendo dal primo elemento ( priorità maggiore ), fino all'ultimo elemento ( priorità minore ). Per mantenere la lista consistente con l’ObserveRelationContainer, alla rimozione di un elemento da quest’ultimo, si rimuove lo stesso dalla sortedObservers.\newline
				\lstinputlisting[language=Java, firstline=176, lastline=179, caption={CoapResource sortedObservers, \href{https://tinyurl.com/y264lheh}{codice originale}},captionpos=b]{../Codice/Java/CoapResource.java}
				\lstinputlisting[language=Java, firstline=809, lastline=831, caption={CoapResource addObserveRelation(...), \href{https://tinyurl.com/y264lheh}{codice originale}},captionpos=b]{../Codice/Java/CoapResource.java}
				\lstinputlisting[language=Java, firstline=840, lastline=852, caption={CoapResource removeObserveRelation(...), \href{https://tinyurl.com/y264lheh}{codice originale}},captionpos=b]{../Codice/Java/CoapResource.java}
				\lstinputlisting[language=Java, firstline=915, lastline=925, caption={CoapResource notifyObserverRelations(...), \href{https://tinyurl.com/y264lheh}{codice originale}},captionpos=b]{../Codice/Java/CoapResource.java}
